\documentclass[11pt]{article}

    \usepackage[breakable]{tcolorbox}
    \usepackage{parskip} % Stop auto-indenting (to mimic markdown behaviour)
    
    \usepackage{iftex}
    \ifPDFTeX
    	\usepackage[T1]{fontenc}
    	\usepackage{mathpazo}
    \else
    	\usepackage{fontspec}
    \fi

    % Basic figure setup, for now with no caption control since it's done
    % automatically by Pandoc (which extracts ![](path) syntax from Markdown).
    \usepackage{graphicx}
    % Maintain compatibility with old templates. Remove in nbconvert 6.0
    \let\Oldincludegraphics\includegraphics
    % Ensure that by default, figures have no caption (until we provide a
    % proper Figure object with a Caption API and a way to capture that
    % in the conversion process - todo).
    \usepackage{caption}
    \DeclareCaptionFormat{nocaption}{}
    \captionsetup{format=nocaption,aboveskip=0pt,belowskip=0pt}

    \usepackage{float}
    \floatplacement{figure}{H} % forces figures to be placed at the correct location
    \usepackage{xcolor} % Allow colors to be defined
    \usepackage{enumerate} % Needed for markdown enumerations to work
    \usepackage{geometry} % Used to adjust the document margins
    \usepackage{amsmath} % Equations
    \usepackage{amssymb} % Equations
    \usepackage{textcomp} % defines textquotesingle
    % Hack from http://tex.stackexchange.com/a/47451/13684:
    \AtBeginDocument{%
        \def\PYZsq{\textquotesingle}% Upright quotes in Pygmentized code
    }
    \usepackage{upquote} % Upright quotes for verbatim code
    \usepackage{eurosym} % defines \euro
    \usepackage[mathletters]{ucs} % Extended unicode (utf-8) support
    \usepackage{fancyvrb} % verbatim replacement that allows latex
    \usepackage{grffile} % extends the file name processing of package graphics 
                         % to support a larger range
    \makeatletter % fix for old versions of grffile with XeLaTeX
    \@ifpackagelater{grffile}{2019/11/01}
    {
      % Do nothing on new versions
    }
    {
      \def\Gread@@xetex#1{%
        \IfFileExists{"\Gin@base".bb}%
        {\Gread@eps{\Gin@base.bb}}%
        {\Gread@@xetex@aux#1}%
      }
    }
    \makeatother
    \usepackage[Export]{adjustbox} % Used to constrain images to a maximum size
    \adjustboxset{max size={0.9\linewidth}{0.9\paperheight}}

    % The hyperref package gives us a pdf with properly built
    % internal navigation ('pdf bookmarks' for the table of contents,
    % internal cross-reference links, web links for URLs, etc.)
    \usepackage{hyperref}
    % The default LaTeX title has an obnoxious amount of whitespace. By default,
    % titling removes some of it. It also provides customization options.
    \usepackage{titling}
    \usepackage{longtable} % longtable support required by pandoc >1.10
    \usepackage{booktabs}  % table support for pandoc > 1.12.2
    \usepackage[inline]{enumitem} % IRkernel/repr support (it uses the enumerate* environment)
    \usepackage[normalem]{ulem} % ulem is needed to support strikethroughs (\sout)
                                % normalem makes italics be italics, not underlines
    \usepackage{mathrsfs}
    

    
    % Colors for the hyperref package
    \definecolor{urlcolor}{rgb}{0,.145,.698}
    \definecolor{linkcolor}{rgb}{.71,0.21,0.01}
    \definecolor{citecolor}{rgb}{.12,.54,.11}

    % ANSI colors
    \definecolor{ansi-black}{HTML}{3E424D}
    \definecolor{ansi-black-intense}{HTML}{282C36}
    \definecolor{ansi-red}{HTML}{E75C58}
    \definecolor{ansi-red-intense}{HTML}{B22B31}
    \definecolor{ansi-green}{HTML}{00A250}
    \definecolor{ansi-green-intense}{HTML}{007427}
    \definecolor{ansi-yellow}{HTML}{DDB62B}
    \definecolor{ansi-yellow-intense}{HTML}{B27D12}
    \definecolor{ansi-blue}{HTML}{208FFB}
    \definecolor{ansi-blue-intense}{HTML}{0065CA}
    \definecolor{ansi-magenta}{HTML}{D160C4}
    \definecolor{ansi-magenta-intense}{HTML}{A03196}
    \definecolor{ansi-cyan}{HTML}{60C6C8}
    \definecolor{ansi-cyan-intense}{HTML}{258F8F}
    \definecolor{ansi-white}{HTML}{C5C1B4}
    \definecolor{ansi-white-intense}{HTML}{A1A6B2}
    \definecolor{ansi-default-inverse-fg}{HTML}{FFFFFF}
    \definecolor{ansi-default-inverse-bg}{HTML}{000000}

    % common color for the border for error outputs.
    \definecolor{outerrorbackground}{HTML}{FFDFDF}

    % commands and environments needed by pandoc snippets
    % extracted from the output of `pandoc -s`
    \providecommand{\tightlist}{%
      \setlength{\itemsep}{0pt}\setlength{\parskip}{0pt}}
    \DefineVerbatimEnvironment{Highlighting}{Verbatim}{commandchars=\\\{\}}
    % Add ',fontsize=\small' for more characters per line
    \newenvironment{Shaded}{}{}
    \newcommand{\KeywordTok}[1]{\textcolor[rgb]{0.00,0.44,0.13}{\textbf{{#1}}}}
    \newcommand{\DataTypeTok}[1]{\textcolor[rgb]{0.56,0.13,0.00}{{#1}}}
    \newcommand{\DecValTok}[1]{\textcolor[rgb]{0.25,0.63,0.44}{{#1}}}
    \newcommand{\BaseNTok}[1]{\textcolor[rgb]{0.25,0.63,0.44}{{#1}}}
    \newcommand{\FloatTok}[1]{\textcolor[rgb]{0.25,0.63,0.44}{{#1}}}
    \newcommand{\CharTok}[1]{\textcolor[rgb]{0.25,0.44,0.63}{{#1}}}
    \newcommand{\StringTok}[1]{\textcolor[rgb]{0.25,0.44,0.63}{{#1}}}
    \newcommand{\CommentTok}[1]{\textcolor[rgb]{0.38,0.63,0.69}{\textit{{#1}}}}
    \newcommand{\OtherTok}[1]{\textcolor[rgb]{0.00,0.44,0.13}{{#1}}}
    \newcommand{\AlertTok}[1]{\textcolor[rgb]{1.00,0.00,0.00}{\textbf{{#1}}}}
    \newcommand{\FunctionTok}[1]{\textcolor[rgb]{0.02,0.16,0.49}{{#1}}}
    \newcommand{\RegionMarkerTok}[1]{{#1}}
    \newcommand{\ErrorTok}[1]{\textcolor[rgb]{1.00,0.00,0.00}{\textbf{{#1}}}}
    \newcommand{\NormalTok}[1]{{#1}}
    
    % Additional commands for more recent versions of Pandoc
    \newcommand{\ConstantTok}[1]{\textcolor[rgb]{0.53,0.00,0.00}{{#1}}}
    \newcommand{\SpecialCharTok}[1]{\textcolor[rgb]{0.25,0.44,0.63}{{#1}}}
    \newcommand{\VerbatimStringTok}[1]{\textcolor[rgb]{0.25,0.44,0.63}{{#1}}}
    \newcommand{\SpecialStringTok}[1]{\textcolor[rgb]{0.73,0.40,0.53}{{#1}}}
    \newcommand{\ImportTok}[1]{{#1}}
    \newcommand{\DocumentationTok}[1]{\textcolor[rgb]{0.73,0.13,0.13}{\textit{{#1}}}}
    \newcommand{\AnnotationTok}[1]{\textcolor[rgb]{0.38,0.63,0.69}{\textbf{\textit{{#1}}}}}
    \newcommand{\CommentVarTok}[1]{\textcolor[rgb]{0.38,0.63,0.69}{\textbf{\textit{{#1}}}}}
    \newcommand{\VariableTok}[1]{\textcolor[rgb]{0.10,0.09,0.49}{{#1}}}
    \newcommand{\ControlFlowTok}[1]{\textcolor[rgb]{0.00,0.44,0.13}{\textbf{{#1}}}}
    \newcommand{\OperatorTok}[1]{\textcolor[rgb]{0.40,0.40,0.40}{{#1}}}
    \newcommand{\BuiltInTok}[1]{{#1}}
    \newcommand{\ExtensionTok}[1]{{#1}}
    \newcommand{\PreprocessorTok}[1]{\textcolor[rgb]{0.74,0.48,0.00}{{#1}}}
    \newcommand{\AttributeTok}[1]{\textcolor[rgb]{0.49,0.56,0.16}{{#1}}}
    \newcommand{\InformationTok}[1]{\textcolor[rgb]{0.38,0.63,0.69}{\textbf{\textit{{#1}}}}}
    \newcommand{\WarningTok}[1]{\textcolor[rgb]{0.38,0.63,0.69}{\textbf{\textit{{#1}}}}}
    
    
    % Define a nice break command that doesn't care if a line doesn't already
    % exist.
    \def\br{\hspace*{\fill} \\* }
    % Math Jax compatibility definitions
    \def\gt{>}
    \def\lt{<}
    \let\Oldtex\TeX
    \let\Oldlatex\LaTeX
    \renewcommand{\TeX}{\textrm{\Oldtex}}
    \renewcommand{\LaTeX}{\textrm{\Oldlatex}}
    % Document parameters
    % Document title
    \title{lec4\_step7\_BarStack\_Aligned\_Stage6}
    
    
    
    
    
% Pygments definitions
\makeatletter
\def\PY@reset{\let\PY@it=\relax \let\PY@bf=\relax%
    \let\PY@ul=\relax \let\PY@tc=\relax%
    \let\PY@bc=\relax \let\PY@ff=\relax}
\def\PY@tok#1{\csname PY@tok@#1\endcsname}
\def\PY@toks#1+{\ifx\relax#1\empty\else%
    \PY@tok{#1}\expandafter\PY@toks\fi}
\def\PY@do#1{\PY@bc{\PY@tc{\PY@ul{%
    \PY@it{\PY@bf{\PY@ff{#1}}}}}}}
\def\PY#1#2{\PY@reset\PY@toks#1+\relax+\PY@do{#2}}

\@namedef{PY@tok@w}{\def\PY@tc##1{\textcolor[rgb]{0.73,0.73,0.73}{##1}}}
\@namedef{PY@tok@c}{\let\PY@it=\textit\def\PY@tc##1{\textcolor[rgb]{0.25,0.50,0.50}{##1}}}
\@namedef{PY@tok@cp}{\def\PY@tc##1{\textcolor[rgb]{0.74,0.48,0.00}{##1}}}
\@namedef{PY@tok@k}{\let\PY@bf=\textbf\def\PY@tc##1{\textcolor[rgb]{0.00,0.50,0.00}{##1}}}
\@namedef{PY@tok@kp}{\def\PY@tc##1{\textcolor[rgb]{0.00,0.50,0.00}{##1}}}
\@namedef{PY@tok@kt}{\def\PY@tc##1{\textcolor[rgb]{0.69,0.00,0.25}{##1}}}
\@namedef{PY@tok@o}{\def\PY@tc##1{\textcolor[rgb]{0.40,0.40,0.40}{##1}}}
\@namedef{PY@tok@ow}{\let\PY@bf=\textbf\def\PY@tc##1{\textcolor[rgb]{0.67,0.13,1.00}{##1}}}
\@namedef{PY@tok@nb}{\def\PY@tc##1{\textcolor[rgb]{0.00,0.50,0.00}{##1}}}
\@namedef{PY@tok@nf}{\def\PY@tc##1{\textcolor[rgb]{0.00,0.00,1.00}{##1}}}
\@namedef{PY@tok@nc}{\let\PY@bf=\textbf\def\PY@tc##1{\textcolor[rgb]{0.00,0.00,1.00}{##1}}}
\@namedef{PY@tok@nn}{\let\PY@bf=\textbf\def\PY@tc##1{\textcolor[rgb]{0.00,0.00,1.00}{##1}}}
\@namedef{PY@tok@ne}{\let\PY@bf=\textbf\def\PY@tc##1{\textcolor[rgb]{0.82,0.25,0.23}{##1}}}
\@namedef{PY@tok@nv}{\def\PY@tc##1{\textcolor[rgb]{0.10,0.09,0.49}{##1}}}
\@namedef{PY@tok@no}{\def\PY@tc##1{\textcolor[rgb]{0.53,0.00,0.00}{##1}}}
\@namedef{PY@tok@nl}{\def\PY@tc##1{\textcolor[rgb]{0.63,0.63,0.00}{##1}}}
\@namedef{PY@tok@ni}{\let\PY@bf=\textbf\def\PY@tc##1{\textcolor[rgb]{0.60,0.60,0.60}{##1}}}
\@namedef{PY@tok@na}{\def\PY@tc##1{\textcolor[rgb]{0.49,0.56,0.16}{##1}}}
\@namedef{PY@tok@nt}{\let\PY@bf=\textbf\def\PY@tc##1{\textcolor[rgb]{0.00,0.50,0.00}{##1}}}
\@namedef{PY@tok@nd}{\def\PY@tc##1{\textcolor[rgb]{0.67,0.13,1.00}{##1}}}
\@namedef{PY@tok@s}{\def\PY@tc##1{\textcolor[rgb]{0.73,0.13,0.13}{##1}}}
\@namedef{PY@tok@sd}{\let\PY@it=\textit\def\PY@tc##1{\textcolor[rgb]{0.73,0.13,0.13}{##1}}}
\@namedef{PY@tok@si}{\let\PY@bf=\textbf\def\PY@tc##1{\textcolor[rgb]{0.73,0.40,0.53}{##1}}}
\@namedef{PY@tok@se}{\let\PY@bf=\textbf\def\PY@tc##1{\textcolor[rgb]{0.73,0.40,0.13}{##1}}}
\@namedef{PY@tok@sr}{\def\PY@tc##1{\textcolor[rgb]{0.73,0.40,0.53}{##1}}}
\@namedef{PY@tok@ss}{\def\PY@tc##1{\textcolor[rgb]{0.10,0.09,0.49}{##1}}}
\@namedef{PY@tok@sx}{\def\PY@tc##1{\textcolor[rgb]{0.00,0.50,0.00}{##1}}}
\@namedef{PY@tok@m}{\def\PY@tc##1{\textcolor[rgb]{0.40,0.40,0.40}{##1}}}
\@namedef{PY@tok@gh}{\let\PY@bf=\textbf\def\PY@tc##1{\textcolor[rgb]{0.00,0.00,0.50}{##1}}}
\@namedef{PY@tok@gu}{\let\PY@bf=\textbf\def\PY@tc##1{\textcolor[rgb]{0.50,0.00,0.50}{##1}}}
\@namedef{PY@tok@gd}{\def\PY@tc##1{\textcolor[rgb]{0.63,0.00,0.00}{##1}}}
\@namedef{PY@tok@gi}{\def\PY@tc##1{\textcolor[rgb]{0.00,0.63,0.00}{##1}}}
\@namedef{PY@tok@gr}{\def\PY@tc##1{\textcolor[rgb]{1.00,0.00,0.00}{##1}}}
\@namedef{PY@tok@ge}{\let\PY@it=\textit}
\@namedef{PY@tok@gs}{\let\PY@bf=\textbf}
\@namedef{PY@tok@gp}{\let\PY@bf=\textbf\def\PY@tc##1{\textcolor[rgb]{0.00,0.00,0.50}{##1}}}
\@namedef{PY@tok@go}{\def\PY@tc##1{\textcolor[rgb]{0.53,0.53,0.53}{##1}}}
\@namedef{PY@tok@gt}{\def\PY@tc##1{\textcolor[rgb]{0.00,0.27,0.87}{##1}}}
\@namedef{PY@tok@err}{\def\PY@bc##1{{\setlength{\fboxsep}{\string -\fboxrule}\fcolorbox[rgb]{1.00,0.00,0.00}{1,1,1}{\strut ##1}}}}
\@namedef{PY@tok@kc}{\let\PY@bf=\textbf\def\PY@tc##1{\textcolor[rgb]{0.00,0.50,0.00}{##1}}}
\@namedef{PY@tok@kd}{\let\PY@bf=\textbf\def\PY@tc##1{\textcolor[rgb]{0.00,0.50,0.00}{##1}}}
\@namedef{PY@tok@kn}{\let\PY@bf=\textbf\def\PY@tc##1{\textcolor[rgb]{0.00,0.50,0.00}{##1}}}
\@namedef{PY@tok@kr}{\let\PY@bf=\textbf\def\PY@tc##1{\textcolor[rgb]{0.00,0.50,0.00}{##1}}}
\@namedef{PY@tok@bp}{\def\PY@tc##1{\textcolor[rgb]{0.00,0.50,0.00}{##1}}}
\@namedef{PY@tok@fm}{\def\PY@tc##1{\textcolor[rgb]{0.00,0.00,1.00}{##1}}}
\@namedef{PY@tok@vc}{\def\PY@tc##1{\textcolor[rgb]{0.10,0.09,0.49}{##1}}}
\@namedef{PY@tok@vg}{\def\PY@tc##1{\textcolor[rgb]{0.10,0.09,0.49}{##1}}}
\@namedef{PY@tok@vi}{\def\PY@tc##1{\textcolor[rgb]{0.10,0.09,0.49}{##1}}}
\@namedef{PY@tok@vm}{\def\PY@tc##1{\textcolor[rgb]{0.10,0.09,0.49}{##1}}}
\@namedef{PY@tok@sa}{\def\PY@tc##1{\textcolor[rgb]{0.73,0.13,0.13}{##1}}}
\@namedef{PY@tok@sb}{\def\PY@tc##1{\textcolor[rgb]{0.73,0.13,0.13}{##1}}}
\@namedef{PY@tok@sc}{\def\PY@tc##1{\textcolor[rgb]{0.73,0.13,0.13}{##1}}}
\@namedef{PY@tok@dl}{\def\PY@tc##1{\textcolor[rgb]{0.73,0.13,0.13}{##1}}}
\@namedef{PY@tok@s2}{\def\PY@tc##1{\textcolor[rgb]{0.73,0.13,0.13}{##1}}}
\@namedef{PY@tok@sh}{\def\PY@tc##1{\textcolor[rgb]{0.73,0.13,0.13}{##1}}}
\@namedef{PY@tok@s1}{\def\PY@tc##1{\textcolor[rgb]{0.73,0.13,0.13}{##1}}}
\@namedef{PY@tok@mb}{\def\PY@tc##1{\textcolor[rgb]{0.40,0.40,0.40}{##1}}}
\@namedef{PY@tok@mf}{\def\PY@tc##1{\textcolor[rgb]{0.40,0.40,0.40}{##1}}}
\@namedef{PY@tok@mh}{\def\PY@tc##1{\textcolor[rgb]{0.40,0.40,0.40}{##1}}}
\@namedef{PY@tok@mi}{\def\PY@tc##1{\textcolor[rgb]{0.40,0.40,0.40}{##1}}}
\@namedef{PY@tok@il}{\def\PY@tc##1{\textcolor[rgb]{0.40,0.40,0.40}{##1}}}
\@namedef{PY@tok@mo}{\def\PY@tc##1{\textcolor[rgb]{0.40,0.40,0.40}{##1}}}
\@namedef{PY@tok@ch}{\let\PY@it=\textit\def\PY@tc##1{\textcolor[rgb]{0.25,0.50,0.50}{##1}}}
\@namedef{PY@tok@cm}{\let\PY@it=\textit\def\PY@tc##1{\textcolor[rgb]{0.25,0.50,0.50}{##1}}}
\@namedef{PY@tok@cpf}{\let\PY@it=\textit\def\PY@tc##1{\textcolor[rgb]{0.25,0.50,0.50}{##1}}}
\@namedef{PY@tok@c1}{\let\PY@it=\textit\def\PY@tc##1{\textcolor[rgb]{0.25,0.50,0.50}{##1}}}
\@namedef{PY@tok@cs}{\let\PY@it=\textit\def\PY@tc##1{\textcolor[rgb]{0.25,0.50,0.50}{##1}}}

\def\PYZbs{\char`\\}
\def\PYZus{\char`\_}
\def\PYZob{\char`\{}
\def\PYZcb{\char`\}}
\def\PYZca{\char`\^}
\def\PYZam{\char`\&}
\def\PYZlt{\char`\<}
\def\PYZgt{\char`\>}
\def\PYZsh{\char`\#}
\def\PYZpc{\char`\%}
\def\PYZdl{\char`\$}
\def\PYZhy{\char`\-}
\def\PYZsq{\char`\'}
\def\PYZdq{\char`\"}
\def\PYZti{\char`\~}
% for compatibility with earlier versions
\def\PYZat{@}
\def\PYZlb{[}
\def\PYZrb{]}
\makeatother


    % For linebreaks inside Verbatim environment from package fancyvrb. 
    \makeatletter
        \newbox\Wrappedcontinuationbox 
        \newbox\Wrappedvisiblespacebox 
        \newcommand*\Wrappedvisiblespace {\textcolor{red}{\textvisiblespace}} 
        \newcommand*\Wrappedcontinuationsymbol {\textcolor{red}{\llap{\tiny$\m@th\hookrightarrow$}}} 
        \newcommand*\Wrappedcontinuationindent {3ex } 
        \newcommand*\Wrappedafterbreak {\kern\Wrappedcontinuationindent\copy\Wrappedcontinuationbox} 
        % Take advantage of the already applied Pygments mark-up to insert 
        % potential linebreaks for TeX processing. 
        %        {, <, #, %, $, ' and ": go to next line. 
        %        _, }, ^, &, >, - and ~: stay at end of broken line. 
        % Use of \textquotesingle for straight quote. 
        \newcommand*\Wrappedbreaksatspecials {% 
            \def\PYGZus{\discretionary{\char`\_}{\Wrappedafterbreak}{\char`\_}}% 
            \def\PYGZob{\discretionary{}{\Wrappedafterbreak\char`\{}{\char`\{}}% 
            \def\PYGZcb{\discretionary{\char`\}}{\Wrappedafterbreak}{\char`\}}}% 
            \def\PYGZca{\discretionary{\char`\^}{\Wrappedafterbreak}{\char`\^}}% 
            \def\PYGZam{\discretionary{\char`\&}{\Wrappedafterbreak}{\char`\&}}% 
            \def\PYGZlt{\discretionary{}{\Wrappedafterbreak\char`\<}{\char`\<}}% 
            \def\PYGZgt{\discretionary{\char`\>}{\Wrappedafterbreak}{\char`\>}}% 
            \def\PYGZsh{\discretionary{}{\Wrappedafterbreak\char`\#}{\char`\#}}% 
            \def\PYGZpc{\discretionary{}{\Wrappedafterbreak\char`\%}{\char`\%}}% 
            \def\PYGZdl{\discretionary{}{\Wrappedafterbreak\char`\$}{\char`\$}}% 
            \def\PYGZhy{\discretionary{\char`\-}{\Wrappedafterbreak}{\char`\-}}% 
            \def\PYGZsq{\discretionary{}{\Wrappedafterbreak\textquotesingle}{\textquotesingle}}% 
            \def\PYGZdq{\discretionary{}{\Wrappedafterbreak\char`\"}{\char`\"}}% 
            \def\PYGZti{\discretionary{\char`\~}{\Wrappedafterbreak}{\char`\~}}% 
        } 
        % Some characters . , ; ? ! / are not pygmentized. 
        % This macro makes them "active" and they will insert potential linebreaks 
        \newcommand*\Wrappedbreaksatpunct {% 
            \lccode`\~`\.\lowercase{\def~}{\discretionary{\hbox{\char`\.}}{\Wrappedafterbreak}{\hbox{\char`\.}}}% 
            \lccode`\~`\,\lowercase{\def~}{\discretionary{\hbox{\char`\,}}{\Wrappedafterbreak}{\hbox{\char`\,}}}% 
            \lccode`\~`\;\lowercase{\def~}{\discretionary{\hbox{\char`\;}}{\Wrappedafterbreak}{\hbox{\char`\;}}}% 
            \lccode`\~`\:\lowercase{\def~}{\discretionary{\hbox{\char`\:}}{\Wrappedafterbreak}{\hbox{\char`\:}}}% 
            \lccode`\~`\?\lowercase{\def~}{\discretionary{\hbox{\char`\?}}{\Wrappedafterbreak}{\hbox{\char`\?}}}% 
            \lccode`\~`\!\lowercase{\def~}{\discretionary{\hbox{\char`\!}}{\Wrappedafterbreak}{\hbox{\char`\!}}}% 
            \lccode`\~`\/\lowercase{\def~}{\discretionary{\hbox{\char`\/}}{\Wrappedafterbreak}{\hbox{\char`\/}}}% 
            \catcode`\.\active
            \catcode`\,\active 
            \catcode`\;\active
            \catcode`\:\active
            \catcode`\?\active
            \catcode`\!\active
            \catcode`\/\active 
            \lccode`\~`\~ 	
        }
    \makeatother

    \let\OriginalVerbatim=\Verbatim
    \makeatletter
    \renewcommand{\Verbatim}[1][1]{%
        %\parskip\z@skip
        \sbox\Wrappedcontinuationbox {\Wrappedcontinuationsymbol}%
        \sbox\Wrappedvisiblespacebox {\FV@SetupFont\Wrappedvisiblespace}%
        \def\FancyVerbFormatLine ##1{\hsize\linewidth
            \vtop{\raggedright\hyphenpenalty\z@\exhyphenpenalty\z@
                \doublehyphendemerits\z@\finalhyphendemerits\z@
                \strut ##1\strut}%
        }%
        % If the linebreak is at a space, the latter will be displayed as visible
        % space at end of first line, and a continuation symbol starts next line.
        % Stretch/shrink are however usually zero for typewriter font.
        \def\FV@Space {%
            \nobreak\hskip\z@ plus\fontdimen3\font minus\fontdimen4\font
            \discretionary{\copy\Wrappedvisiblespacebox}{\Wrappedafterbreak}
            {\kern\fontdimen2\font}%
        }%
        
        % Allow breaks at special characters using \PYG... macros.
        \Wrappedbreaksatspecials
        % Breaks at punctuation characters . , ; ? ! and / need catcode=\active 	
        \OriginalVerbatim[#1,codes*=\Wrappedbreaksatpunct]%
    }
    \makeatother

    % Exact colors from NB
    \definecolor{incolor}{HTML}{303F9F}
    \definecolor{outcolor}{HTML}{D84315}
    \definecolor{cellborder}{HTML}{CFCFCF}
    \definecolor{cellbackground}{HTML}{F7F7F7}
    
    % prompt
    \makeatletter
    \newcommand{\boxspacing}{\kern\kvtcb@left@rule\kern\kvtcb@boxsep}
    \makeatother
    \newcommand{\prompt}[4]{
        {\ttfamily\llap{{\color{#2}[#3]:\hspace{3pt}#4}}\vspace{-\baselineskip}}
    }
    

    
    % Prevent overflowing lines due to hard-to-break entities
    \sloppy 
    % Setup hyperref package
    \hypersetup{
      breaklinks=true,  % so long urls are correctly broken across lines
      colorlinks=true,
      urlcolor=urlcolor,
      linkcolor=linkcolor,
      citecolor=citecolor,
      }
    % Slightly bigger margins than the latex defaults
    
    \geometry{verbose,tmargin=1in,bmargin=1in,lmargin=1in,rmargin=1in}
    
    

\begin{document}
    
    \maketitle
    
    

    
    \begin{tcolorbox}[breakable, size=fbox, boxrule=1pt, pad at break*=1mm,colback=cellbackground, colframe=cellborder]
\prompt{In}{incolor}{ }{\boxspacing}
\begin{Verbatim}[commandchars=\\\{\}]
\PY{c+c1}{\PYZsh{}\PYZsh{} Python basics for novice data scientists, supported by Wagatsuma Lab@Kyutech }
\PY{c+c1}{\PYZsh{}}
\PY{c+c1}{\PYZsh{} The MIT License (MIT): Copyright (c) 2020 Hiroaki Wagatsuma and Wagatsuma Lab@Kyutech}
\PY{c+c1}{\PYZsh{} }
\PY{c+c1}{\PYZsh{} Permission is hereby granted, free of charge, to any person obtaining a copy of this software and associated documentation files (the \PYZdq{}Software\PYZdq{}), to deal in the Software without restriction, including without limitation the rights to use, copy, modify, merge, publish, distribute, sublicense, and/or sell copies of the Software, and to permit persons to whom the Software is furnished to do so, subject to the following conditions:}
\PY{c+c1}{\PYZsh{} The above copyright notice and this permission notice shall be included in all copies or substantial portions of the Software.}
\PY{c+c1}{\PYZsh{} THE SOFTWARE IS PROVIDED \PYZdq{}AS IS\PYZdq{}, WITHOUT WARRANTY OF ANY KIND, EXPRESS OR IMPLIED, INCLUDING BUT NOT LIMITED TO THE WARRANTIES OF MERCHANTABILITY, FITNESS FOR A PARTICULAR PURPOSE AND NONINFRINGEMENT. IN NO EVENT SHALL THE AUTHORS OR COPYRIGHT HOLDERS BE LIABLE FOR ANY CLAIM, DAMAGES OR OTHER LIABILITY, WHETHER IN AN ACTION OF CONTRACT, TORT OR OTHERWISE, ARISING FROM, OUT OF OR IN CONNECTION WITH THE SOFTWARE OR THE USE OR OTHER DEALINGS IN THE SOFTWARE. */}
\PY{c+c1}{\PYZsh{}}
\PY{c+c1}{\PYZsh{} \PYZsh{} @Time    : 2020\PYZhy{}11\PYZhy{}30 }
\PY{c+c1}{\PYZsh{} \PYZsh{} @Author  : Hiroaki Wagatsuma}
\PY{c+c1}{\PYZsh{} \PYZsh{} @Site    : https://github.com/hirowgit/2A1\PYZus{}python\PYZus{}intermediate\PYZus{}course}
\PY{c+c1}{\PYZsh{} \PYZsh{} @IDE     : Python 3.9.14 (main, Sep  6 2022, 23:29:09) [Clang 13.1.6 (clang\PYZhy{}1316.0.21.2.5)] on darwin}
\PY{c+c1}{\PYZsh{} \PYZsh{} @File    : lec4\PYZus{}step7\PYZus{}BarStack\PYZus{}Aligned\PYZus{}Stage6.py }
\end{Verbatim}
\end{tcolorbox}

    \begin{tcolorbox}[breakable, size=fbox, boxrule=1pt, pad at break*=1mm,colback=cellbackground, colframe=cellborder]
\prompt{In}{incolor}{292}{\boxspacing}
\begin{Verbatim}[commandchars=\\\{\}]
\PY{k+kn}{import} \PY{n+nn}{numpy} \PY{k}{as} \PY{n+nn}{np}
\PY{c+c1}{\PYZsh{}prFill=[90    60    50    50    50    90    40    30    80    40  20 ]/100;}
\PY{n}{prFill}\PY{o}{=}\PY{n}{np}\PY{o}{.}\PY{n}{array}\PY{p}{(}\PY{p}{[}\PY{l+m+mi}{90}\PY{p}{,} \PY{l+m+mi}{60}\PY{p}{,} \PY{l+m+mi}{50}\PY{p}{,} \PY{l+m+mi}{50}\PY{p}{,} \PY{l+m+mi}{50}\PY{p}{,} \PY{l+m+mi}{90}\PY{p}{,} \PY{l+m+mi}{40}\PY{p}{,} \PY{l+m+mi}{30}\PY{p}{,} \PY{l+m+mi}{80}\PY{p}{,} \PY{l+m+mi}{40}\PY{p}{,} \PY{l+m+mi}{20}\PY{p}{]}\PY{p}{)}
\PY{n}{prFill}\PY{o}{=}\PY{n}{prFill}\PY{o}{/}\PY{l+m+mi}{100}
\PY{n}{fillLine}\PY{o}{=}\PY{n}{np}\PY{o}{.}\PY{n}{full}\PY{p}{(}\PY{n+nb}{len}\PY{p}{(}\PY{n}{prFill}\PY{p}{)}\PY{p}{,}\PY{k+kc}{True}\PY{p}{)}
\PY{n}{LineT}\PY{o}{=}\PY{p}{[}\PY{p}{]}
\PY{n}{tmp}\PY{o}{=}\PY{p}{[}\PY{p}{]}
\PY{n}{k}\PY{o}{=}\PY{l+m+mi}{0}
\PY{k}{for} \PY{n}{i} \PY{o+ow}{in} \PY{n+nb}{range}\PY{p}{(}\PY{n+nb}{len}\PY{p}{(}\PY{n}{prFill}\PY{p}{)}\PY{p}{)}\PY{p}{:}
\PY{c+c1}{\PYZsh{}for i in range(5):}
\PY{c+c1}{\PYZsh{}for i in range(5):}
    \PY{k}{if} \PY{n}{fillLine}\PY{p}{[}\PY{n}{i}\PY{p}{]}\PY{p}{:}
        \PY{n}{remF}\PY{o}{=}\PY{l+m+mi}{1}\PY{o}{\PYZhy{}}\PY{n}{prFill}\PY{p}{[}\PY{n}{i}\PY{p}{]}
        \PY{n}{IDrem}\PY{o}{=}\PY{n}{np}\PY{o}{.}\PY{n}{where}\PY{p}{(}\PY{p}{(}\PY{n}{prFill}\PY{p}{[}\PY{n}{i}\PY{o}{+}\PY{l+m+mi}{1}\PY{p}{:}\PY{n+nb}{len}\PY{p}{(}\PY{n}{prFill}\PY{p}{)}\PY{p}{]}\PY{o}{\PYZlt{}}\PY{o}{=}\PY{n}{remF}\PY{p}{)} \PY{o}{\PYZam{}} \PY{n}{fillLine}\PY{p}{[}\PY{n}{i}\PY{o}{+}\PY{l+m+mi}{1}\PY{p}{:}\PY{n+nb}{len}\PY{p}{(}\PY{n}{prFill}\PY{p}{)}\PY{p}{]}\PY{p}{)}
        \PY{n}{tmp}\PY{o}{=}\PY{n}{i}
        \PY{n}{fID}\PY{o}{=}\PY{n}{i}
        \PY{c+c1}{\PYZsh{}j=0}
        \PY{k}{while} \PY{n}{IDrem}\PY{p}{[}\PY{l+m+mi}{0}\PY{p}{]}\PY{o}{.}\PY{n}{size} \PY{o}{\PYZgt{}} \PY{l+m+mi}{0}\PY{p}{:}
            \PY{n}{fID}\PY{o}{=}\PY{n}{IDrem}\PY{p}{[}\PY{l+m+mi}{0}\PY{p}{]}\PY{p}{[}\PY{l+m+mi}{0}\PY{p}{]}\PY{o}{+}\PY{n}{fID}\PY{o}{+}\PY{l+m+mi}{1}
            \PY{n}{tmp}\PY{o}{=}\PY{n}{np}\PY{o}{.}\PY{n}{append}\PY{p}{(}\PY{n}{tmp}\PY{p}{,}\PY{n}{fID}\PY{p}{)}
            \PY{n}{remF}\PY{o}{=}\PY{n}{remF}\PY{o}{\PYZhy{}}\PY{n}{prFill}\PY{p}{[}\PY{n}{fID}\PY{p}{]}
            \PY{n}{IDrem}\PY{o}{=}\PY{n}{np}\PY{o}{.}\PY{n}{where}\PY{p}{(}\PY{p}{(}\PY{n}{prFill}\PY{p}{[}\PY{n}{fID}\PY{o}{+}\PY{l+m+mi}{1}\PY{p}{:}\PY{n+nb}{len}\PY{p}{(}\PY{n}{prFill}\PY{p}{)}\PY{p}{]}\PY{o}{\PYZlt{}}\PY{o}{=}\PY{n}{remF}\PY{p}{)} \PY{o}{\PYZam{}} \PY{n}{fillLine}\PY{p}{[}\PY{n}{fID}\PY{o}{+}\PY{l+m+mi}{1}\PY{p}{:}\PY{n+nb}{len}\PY{p}{(}\PY{n}{prFill}\PY{p}{)}\PY{p}{]}\PY{p}{)}
        \PY{n}{LineT}\PY{o}{.}\PY{n}{append}\PY{p}{(}\PY{n}{tmp}\PY{p}{)}
        \PY{n}{fillLine}\PY{p}{[}\PY{n}{tmp}\PY{p}{]}\PY{o}{=}\PY{k+kc}{False}
        \PY{n+nb}{print}\PY{p}{(}\PY{l+s+s2}{\PYZdq{}}\PY{l+s+s2}{k;}\PY{l+s+s2}{\PYZdq{}}\PY{p}{,}\PY{n}{k}\PY{p}{)}
        \PY{n+nb}{print}\PY{p}{(}\PY{l+s+s2}{\PYZdq{}}\PY{l+s+s2}{LineT;}\PY{l+s+s2}{\PYZdq{}}\PY{p}{,}\PY{n}{LineT}\PY{p}{)}
        \PY{n}{k}\PY{o}{=}\PY{n}{k}\PY{o}{+}\PY{l+m+mi}{1}
        \PY{n+nb}{print}\PY{p}{(}\PY{l+s+s2}{\PYZdq{}}\PY{l+s+s2}{k;}\PY{l+s+s2}{\PYZdq{}}\PY{p}{,}\PY{n}{k}\PY{p}{)}
\end{Verbatim}
\end{tcolorbox}

    \begin{Verbatim}[commandchars=\\\{\}]
k; 0
LineT; [0]
k; 1
k; 1
LineT; [0, array([1, 6], dtype=int64)]
k; 2
k; 2
LineT; [0, array([1, 6], dtype=int64), array([2, 3], dtype=int64)]
k; 3
k; 3
LineT; [0, array([1, 6], dtype=int64), array([2, 3], dtype=int64), array([ 4,
7, 10], dtype=int64)]
k; 4
k; 4
LineT; [0, array([1, 6], dtype=int64), array([2, 3], dtype=int64), array([ 4,
7, 10], dtype=int64), 5]
k; 5
k; 5
LineT; [0, array([1, 6], dtype=int64), array([2, 3], dtype=int64), array([ 4,
7, 10], dtype=int64), 5, 8]
k; 6
k; 6
LineT; [0, array([1, 6], dtype=int64), array([2, 3], dtype=int64), array([ 4,
7, 10], dtype=int64), 5, 8, 9]
k; 7
    \end{Verbatim}

    \begin{tcolorbox}[breakable, size=fbox, boxrule=1pt, pad at break*=1mm,colback=cellbackground, colframe=cellborder]
\prompt{In}{incolor}{293}{\boxspacing}
\begin{Verbatim}[commandchars=\\\{\}]
\PY{c+c1}{\PYZsh{}行列の要素内の型が違うことが問題となる}
\PY{k}{for} \PY{n}{i} \PY{o+ow}{in} \PY{n}{LineT}\PY{p}{:}
    \PY{n+nb}{print}\PY{p}{(}\PY{n+nb}{type}\PY{p}{(}\PY{n}{i}\PY{p}{)}\PY{p}{)}
\end{Verbatim}
\end{tcolorbox}

    \begin{Verbatim}[commandchars=\\\{\}]
<class 'int'>
<class 'numpy.ndarray'>
<class 'numpy.ndarray'>
<class 'numpy.ndarray'>
<class 'int'>
<class 'int'>
<class 'int'>
    \end{Verbatim}

    \begin{tcolorbox}[breakable, size=fbox, boxrule=1pt, pad at break*=1mm,colback=cellbackground, colframe=cellborder]
\prompt{In}{incolor}{294}{\boxspacing}
\begin{Verbatim}[commandchars=\\\{\}]
\PY{c+c1}{\PYZsh{}型がintのものを\PYZsq{}numpy.ndarray\PYZsq{}に変換}
\PY{n}{f\PYZus{}LineT} \PY{o}{=} \PY{p}{[}\PY{n}{np}\PY{o}{.}\PY{n}{array}\PY{p}{(}\PY{n}{i}\PY{p}{)} \PY{k}{if} \PY{n+nb}{type}\PY{p}{(}\PY{n}{i}\PY{p}{)}\PY{o}{==}\PY{n+nb}{int} \PY{k}{else} \PY{n}{i} \PY{k}{for} \PY{n}{i} \PY{o+ow}{in} \PY{n}{LineT}\PY{p}{]}
\PY{n+nb}{print}\PY{p}{(}\PY{n}{f\PYZus{}LineT}\PY{p}{)}
\PY{p}{[}\PY{n+nb}{print}\PY{p}{(}\PY{n+nb}{type}\PY{p}{(}\PY{n}{i}\PY{p}{)}\PY{p}{)} \PY{k}{for} \PY{n}{i} \PY{o+ow}{in} \PY{n}{f\PYZus{}LineT}\PY{p}{]}
\end{Verbatim}
\end{tcolorbox}

    \begin{Verbatim}[commandchars=\\\{\}]
[array(0), array([1, 6], dtype=int64), array([2, 3], dtype=int64), array([ 4,
7, 10], dtype=int64), array(5), array(8), array(9)]
<class 'numpy.ndarray'>
<class 'numpy.ndarray'>
<class 'numpy.ndarray'>
<class 'numpy.ndarray'>
<class 'numpy.ndarray'>
<class 'numpy.ndarray'>
<class 'numpy.ndarray'>
    \end{Verbatim}

            \begin{tcolorbox}[breakable, size=fbox, boxrule=.5pt, pad at break*=1mm, opacityfill=0]
\prompt{Out}{outcolor}{294}{\boxspacing}
\begin{Verbatim}[commandchars=\\\{\}]
[None, None, None, None, None, None, None]
\end{Verbatim}
\end{tcolorbox}
        
    \begin{tcolorbox}[breakable, size=fbox, boxrule=1pt, pad at break*=1mm,colback=cellbackground, colframe=cellborder]
\prompt{In}{incolor}{295}{\boxspacing}
\begin{Verbatim}[commandchars=\\\{\}]
\PY{n}{LineT}
\end{Verbatim}
\end{tcolorbox}

            \begin{tcolorbox}[breakable, size=fbox, boxrule=.5pt, pad at break*=1mm, opacityfill=0]
\prompt{Out}{outcolor}{295}{\boxspacing}
\begin{Verbatim}[commandchars=\\\{\}]
[0,
 array([1, 6], dtype=int64),
 array([2, 3], dtype=int64),
 array([ 4,  7, 10], dtype=int64),
 5,
 8,
 9]
\end{Verbatim}
\end{tcolorbox}
        
    \begin{tcolorbox}[breakable, size=fbox, boxrule=1pt, pad at break*=1mm,colback=cellbackground, colframe=cellborder]
\prompt{In}{incolor}{296}{\boxspacing}
\begin{Verbatim}[commandchars=\\\{\}]
\PY{c+c1}{\PYZsh{}lenLineT=cell2mat(cellfun(@(x) length(x),LineT,\PYZsq{}UniformOutput\PYZsq{},false));に該当部}
\PY{n}{lenLineT} \PY{o}{=} \PY{p}{[}\PY{n}{i}\PY{o}{.}\PY{n}{size} \PY{k}{for} \PY{n}{i} \PY{o+ow}{in} \PY{n}{f\PYZus{}LineT}\PY{p}{]}
\PY{n+nb}{print}\PY{p}{(}\PY{n}{lenLineT}\PY{p}{)}
\end{Verbatim}
\end{tcolorbox}

    \begin{Verbatim}[commandchars=\\\{\}]
[1, 2, 2, 3, 1, 1, 1]
    \end{Verbatim}

    \begin{tcolorbox}[breakable, size=fbox, boxrule=1pt, pad at break*=1mm,colback=cellbackground, colframe=cellborder]
\prompt{In}{incolor}{297}{\boxspacing}
\begin{Verbatim}[commandchars=\\\{\}]
\PY{c+c1}{\PYZsh{}stackBarD=zeros(size(LineT,2),max(lenLineT));に該当部}
\PY{n}{stackBarD} \PY{o}{=} \PY{n}{np}\PY{o}{.}\PY{n}{zeros}\PY{p}{(}\PY{p}{(}\PY{n}{np}\PY{o}{.}\PY{n}{shape}\PY{p}{(}\PY{n}{f\PYZus{}LineT}\PY{p}{)}\PY{p}{[}\PY{l+m+mi}{0}\PY{p}{]}\PY{p}{,}\PY{n+nb}{max}\PY{p}{(}\PY{n}{lenLineT}\PY{p}{)}\PY{p}{)}\PY{p}{)}
\end{Verbatim}
\end{tcolorbox}

    \begin{Verbatim}[commandchars=\\\{\}]
[[0. 0. 0.]
 [0. 0. 0.]
 [0. 0. 0.]
 [0. 0. 0.]
 [0. 0. 0.]
 [0. 0. 0.]
 [0. 0. 0.]]
    \end{Verbatim}

    \begin{Verbatim}[commandchars=\\\{\}]
C:\textbackslash{}Users\textbackslash{}Kaito\textbackslash{}anaconda3\textbackslash{}lib\textbackslash{}site-packages\textbackslash{}numpy\textbackslash{}core\textbackslash{}\_asarray.py:102:
VisibleDeprecationWarning: Creating an ndarray from ragged nested sequences
(which is a list-or-tuple of lists-or-tuples-or ndarrays with different lengths
or shapes) is deprecated. If you meant to do this, you must specify
'dtype=object' when creating the ndarray.
  return array(a, dtype, copy=False, order=order)
    \end{Verbatim}

    \begin{tcolorbox}[breakable, size=fbox, boxrule=1pt, pad at break*=1mm,colback=cellbackground, colframe=cellborder]
\prompt{In}{incolor}{298}{\boxspacing}
\begin{Verbatim}[commandchars=\\\{\}]
\PY{n+nb}{len}\PY{p}{(}\PY{n}{f\PYZus{}LineT}\PY{p}{)}
\end{Verbatim}
\end{tcolorbox}

            \begin{tcolorbox}[breakable, size=fbox, boxrule=.5pt, pad at break*=1mm, opacityfill=0]
\prompt{Out}{outcolor}{298}{\boxspacing}
\begin{Verbatim}[commandchars=\\\{\}]
7
\end{Verbatim}
\end{tcolorbox}
        
    \begin{tcolorbox}[breakable, size=fbox, boxrule=1pt, pad at break*=1mm,colback=cellbackground, colframe=cellborder]
\prompt{In}{incolor}{299}{\boxspacing}
\begin{Verbatim}[commandchars=\\\{\}]
\PY{n}{f\PYZus{}LineT}\PY{p}{[}\PY{l+m+mi}{1}\PY{p}{]}
\PY{n+nb}{print}\PY{p}{(}\PY{n}{prFill}\PY{p}{)}
\PY{n}{i} \PY{o}{=} \PY{p}{[}\PY{l+m+mi}{1}\PY{p}{,} \PY{l+m+mi}{6}\PY{p}{]}
\PY{n+nb}{print}\PY{p}{(}\PY{n}{prFill}\PY{p}{[}\PY{n}{i}\PY{p}{]}\PY{p}{)}
\end{Verbatim}
\end{tcolorbox}

    \begin{Verbatim}[commandchars=\\\{\}]
[0.9 0.6 0.5 0.5 0.5 0.9 0.4 0.3 0.8 0.4 0.2]
[0.6 0.4]
    \end{Verbatim}

    \begin{tcolorbox}[breakable, size=fbox, boxrule=1pt, pad at break*=1mm,colback=cellbackground, colframe=cellborder]
\prompt{In}{incolor}{300}{\boxspacing}
\begin{Verbatim}[commandchars=\\\{\}]
\PY{c+c1}{\PYZsh{}stackBarDにグラフに積み上げるprFillの値を置き換え}
\PY{k}{for} \PY{n}{i} \PY{o+ow}{in} \PY{n+nb}{range}\PY{p}{(}\PY{n+nb}{len}\PY{p}{(}\PY{n}{f\PYZus{}LineT}\PY{p}{)}\PY{p}{)}\PY{p}{:}
    \PY{n}{tmp} \PY{o}{=} \PY{n}{f\PYZus{}LineT}\PY{p}{[}\PY{n}{i}\PY{p}{]}
    \PY{n}{stackBarD}\PY{p}{[}\PY{n}{i}\PY{p}{,}\PY{l+m+mi}{0}\PY{p}{:}\PY{n}{lenLineT}\PY{p}{[}\PY{n}{i}\PY{p}{]}\PY{p}{]} \PY{o}{=} \PY{n}{prFill}\PY{p}{[}\PY{n}{tmp}\PY{p}{]}
\PY{n+nb}{print}\PY{p}{(}\PY{n}{stackBarD}\PY{p}{)}
\end{Verbatim}
\end{tcolorbox}

    \begin{Verbatim}[commandchars=\\\{\}]
1
[0.]
[[0.9 0.  0. ]
 [0.  0.  0. ]
 [0.  0.  0. ]
 [0.  0.  0. ]
 [0.  0.  0. ]
 [0.  0.  0. ]
 [0.  0.  0. ]]
2
[0. 0.]
[[0.9 0.  0. ]
 [0.6 0.4 0. ]
 [0.  0.  0. ]
 [0.  0.  0. ]
 [0.  0.  0. ]
 [0.  0.  0. ]
 [0.  0.  0. ]]
2
[0. 0.]
[[0.9 0.  0. ]
 [0.6 0.4 0. ]
 [0.5 0.5 0. ]
 [0.  0.  0. ]
 [0.  0.  0. ]
 [0.  0.  0. ]
 [0.  0.  0. ]]
3
[0. 0. 0.]
[[0.9 0.  0. ]
 [0.6 0.4 0. ]
 [0.5 0.5 0. ]
 [0.5 0.3 0.2]
 [0.  0.  0. ]
 [0.  0.  0. ]
 [0.  0.  0. ]]
1
[0.]
[[0.9 0.  0. ]
 [0.6 0.4 0. ]
 [0.5 0.5 0. ]
 [0.5 0.3 0.2]
 [0.9 0.  0. ]
 [0.  0.  0. ]
 [0.  0.  0. ]]
1
[0.]
[[0.9 0.  0. ]
 [0.6 0.4 0. ]
 [0.5 0.5 0. ]
 [0.5 0.3 0.2]
 [0.9 0.  0. ]
 [0.8 0.  0. ]
 [0.  0.  0. ]]
1
[0.]
[[0.9 0.  0. ]
 [0.6 0.4 0. ]
 [0.5 0.5 0. ]
 [0.5 0.3 0.2]
 [0.9 0.  0. ]
 [0.8 0.  0. ]
 [0.4 0.  0. ]]
    \end{Verbatim}

    \begin{tcolorbox}[breakable, size=fbox, boxrule=1pt, pad at break*=1mm,colback=cellbackground, colframe=cellborder]
\prompt{In}{incolor}{ }{\boxspacing}
\begin{Verbatim}[commandchars=\\\{\}]

\end{Verbatim}
\end{tcolorbox}

    \begin{tcolorbox}[breakable, size=fbox, boxrule=1pt, pad at break*=1mm,colback=cellbackground, colframe=cellborder]
\prompt{In}{incolor}{301}{\boxspacing}
\begin{Verbatim}[commandchars=\\\{\}]
\PY{n}{y\PYZus{}data\PYZus{}stack} \PY{o}{=} \PY{p}{[}\PY{p}{]}
\PY{n}{y\PYZus{}data\PYZus{}stack} \PY{o}{=} \PY{n+nb}{tuple}\PY{p}{(}\PY{p}{[}\PY{n}{np}\PY{o}{.}\PY{n}{append}\PY{p}{(}\PY{n}{y\PYZus{}data\PYZus{}stack}\PY{p}{,} \PY{n}{i}\PY{p}{)}  \PY{k}{for} \PY{n}{i} \PY{o+ow}{in} \PY{n}{stackBarD}\PY{p}{]}\PY{p}{)}
\PY{n+nb}{print}\PY{p}{(}\PY{n}{y\PYZus{}data\PYZus{}stack}\PY{p}{)}
\end{Verbatim}
\end{tcolorbox}

    \begin{Verbatim}[commandchars=\\\{\}]
(array([0.9, 0. , 0. ]), array([0.6, 0.4, 0. ]), array([0.5, 0.5, 0. ]),
array([0.5, 0.3, 0.2]), array([0.9, 0. , 0. ]), array([0.8, 0. , 0. ]),
array([0.4, 0. , 0. ]))
    \end{Verbatim}

    \begin{tcolorbox}[breakable, size=fbox, boxrule=1pt, pad at break*=1mm,colback=cellbackground, colframe=cellborder]
\prompt{In}{incolor}{302}{\boxspacing}
\begin{Verbatim}[commandchars=\\\{\}]
\PY{n}{LineT}\PY{p}{[}\PY{l+m+mi}{0}\PY{p}{]}
\end{Verbatim}
\end{tcolorbox}

            \begin{tcolorbox}[breakable, size=fbox, boxrule=.5pt, pad at break*=1mm, opacityfill=0]
\prompt{Out}{outcolor}{302}{\boxspacing}
\begin{Verbatim}[commandchars=\\\{\}]
0
\end{Verbatim}
\end{tcolorbox}
        
    \begin{tcolorbox}[breakable, size=fbox, boxrule=1pt, pad at break*=1mm,colback=cellbackground, colframe=cellborder]
\prompt{In}{incolor}{331}{\boxspacing}
\begin{Verbatim}[commandchars=\\\{\}]
\PY{n}{x\PYZus{}label} \PY{o}{=} \PY{p}{[}\PY{n}{i}\PY{o}{+}\PY{l+m+mi}{1} \PY{k}{for} \PY{n}{i} \PY{o+ow}{in} \PY{n+nb}{range}\PY{p}{(}\PY{n+nb}{len}\PY{p}{(}\PY{n}{prFill}\PY{p}{)}\PY{p}{)}\PY{p}{]}
\PY{n}{x\PYZus{}stack\PYZus{}label} \PY{o}{=} \PY{p}{[}\PY{n}{i}\PY{o}{+}\PY{l+m+mi}{1} \PY{k}{for} \PY{n}{i} \PY{o+ow}{in} \PY{n+nb}{range}\PY{p}{(}\PY{n+nb}{len}\PY{p}{(}\PY{n}{stackBarD}\PY{p}{)}\PY{p}{)}\PY{p}{]}
\PY{n}{y\PYZus{}label} \PY{o}{=} \PY{n}{np}\PY{o}{.}\PY{n}{arange}\PY{p}{(}\PY{l+m+mi}{0}\PY{p}{,} \PY{l+m+mi}{12}\PY{p}{,} \PY{l+m+mi}{2}\PY{p}{)}
\PY{n}{y\PYZus{}label} \PY{o}{=}\PY{p}{[}\PY{n}{i}\PY{o}{/}\PY{l+m+mi}{10} \PY{k}{for} \PY{n}{i} \PY{o+ow}{in} \PY{n}{y\PYZus{}label}\PY{p}{]}
\PY{n+nb}{print}\PY{p}{(}\PY{n}{x\PYZus{}label}\PY{p}{)}
\PY{n+nb}{print}\PY{p}{(}\PY{n}{x\PYZus{}stack\PYZus{}label}\PY{p}{)}
\PY{n+nb}{print}\PY{p}{(}\PY{n}{y\PYZus{}label}\PY{p}{)}
\end{Verbatim}
\end{tcolorbox}

    \begin{Verbatim}[commandchars=\\\{\}]
[1, 2, 3, 4, 5, 6, 7, 8, 9, 10, 11]
[1, 2, 3, 4, 5, 6, 7]
[0.0, 0.2, 0.4, 0.6, 0.8, 1.0]
    \end{Verbatim}

    \begin{tcolorbox}[breakable, size=fbox, boxrule=1pt, pad at break*=1mm,colback=cellbackground, colframe=cellborder]
\prompt{In}{incolor}{332}{\boxspacing}
\begin{Verbatim}[commandchars=\\\{\}]
\PY{c+c1}{\PYZsh{}matplotlibモジュールの読み込み}
\PY{k+kn}{import} \PY{n+nn}{matplotlib}\PY{n+nn}{.}\PY{n+nn}{pyplot} \PY{k}{as} \PY{n+nn}{plt}
\end{Verbatim}
\end{tcolorbox}

    \begin{tcolorbox}[breakable, size=fbox, boxrule=1pt, pad at break*=1mm,colback=cellbackground, colframe=cellborder]
\prompt{In}{incolor}{336}{\boxspacing}
\begin{Verbatim}[commandchars=\\\{\}]
\PY{c+c1}{\PYZsh{}2つのグラフの表示画面の分割}
\PY{n}{fig} \PY{o}{=} \PY{n}{plt}\PY{o}{.}\PY{n}{figure}\PY{p}{(}\PY{n}{figsize}\PY{o}{=}\PY{p}{(}\PY{l+m+mi}{30}\PY{p}{,}\PY{l+m+mi}{20}\PY{p}{)}\PY{p}{,} \PY{n}{dpi}\PY{o}{=}\PY{l+m+mi}{50}\PY{p}{)}
\PY{n}{init\PYZus{}fig} \PY{o}{=} \PY{n}{fig}\PY{o}{.}\PY{n}{add\PYZus{}subplot}\PY{p}{(}\PY{l+m+mi}{2} \PY{p}{,} \PY{l+m+mi}{1}\PY{p}{,} \PY{l+m+mi}{1}\PY{p}{)}
\PY{n}{stack\PYZus{}fig} \PY{o}{=} \PY{n}{fig}\PY{o}{.}\PY{n}{add\PYZus{}subplot}\PY{p}{(}\PY{l+m+mi}{2}\PY{p}{,} \PY{l+m+mi}{1}\PY{p}{,} \PY{l+m+mi}{2}\PY{p}{)}

\PY{c+c1}{\PYZsh{}上のグラフの表示設定}
\PY{c+c1}{\PYZsh{}参考;https://www.yutaka\PYZhy{}note.com/entry/matplotlib\PYZus{}axis}

\PY{n}{init\PYZus{}fig}\PY{o}{.}\PY{n}{set\PYZus{}xlabel}\PY{p}{(}\PY{l+s+s2}{\PYZdq{}}\PY{l+s+s2}{store ID}\PY{l+s+s2}{\PYZdq{}}\PY{p}{,} \PY{n}{size} \PY{o}{=} \PY{l+m+mi}{25}\PY{p}{)}
\PY{n}{init\PYZus{}fig}\PY{o}{.}\PY{n}{set\PYZus{}xticks}\PY{p}{(}\PY{n}{x\PYZus{}label}\PY{p}{)}
\PY{n}{init\PYZus{}fig}\PY{o}{.}\PY{n}{set\PYZus{}xticklabels}\PY{p}{(}\PY{n}{x\PYZus{}label}\PY{p}{,} \PY{n}{size}\PY{o}{=}\PY{l+m+mi}{20}\PY{p}{)}

\PY{n}{init\PYZus{}fig}\PY{o}{.}\PY{n}{set\PYZus{}ylabel}\PY{p}{(}\PY{l+s+s2}{\PYZdq{}}\PY{l+s+s2}{Action Steps(AS)}\PY{l+s+s2}{\PYZdq{}}\PY{p}{,} \PY{n}{size} \PY{o}{=} \PY{l+m+mi}{25}\PY{p}{)}
\PY{n}{init\PYZus{}fig}\PY{o}{.}\PY{n}{set\PYZus{}yticks}\PY{p}{(}\PY{n}{y\PYZus{}label}\PY{p}{)}
\PY{n}{init\PYZus{}fig}\PY{o}{.}\PY{n}{set\PYZus{}yticklabels}\PY{p}{(}\PY{n}{y\PYZus{}label}\PY{p}{,} \PY{n}{size}\PY{o}{=}\PY{l+m+mi}{20}\PY{p}{)}
\PY{n}{init\PYZus{}fig}\PY{o}{.}\PY{n}{set\PYZus{}ylim}\PY{p}{(}\PY{l+m+mi}{0} \PY{p}{,} \PY{l+m+mi}{1}\PY{p}{)}
\PY{c+c1}{\PYZsh{} init\PYZus{}fig.set\PYZus{}yticks(np.arange(0, 1, 0.2))}
\PY{c+c1}{\PYZsh{} init\PYZus{}fig.title(\PYZdq{}\PYZdq{})}
\PY{n}{init\PYZus{}fig}\PY{o}{.}\PY{n}{grid}\PY{p}{(}\PY{k+kc}{True}\PY{p}{)}
\PY{c+c1}{\PYZsh{}上のグラフの表示}
\PY{n}{bar} \PY{o}{=} \PY{n}{init\PYZus{}fig}\PY{o}{.}\PY{n}{bar}\PY{p}{(}\PY{n}{x\PYZus{}label}\PY{p}{,} \PY{n}{prFill}\PY{p}{,} \PY{n}{color} \PY{o}{=} \PY{l+s+s1}{\PYZsq{}}\PY{l+s+s1}{w}\PY{l+s+s1}{\PYZsq{}}\PY{p}{,} \PY{n}{edgecolor} \PY{o}{=}\PY{l+s+s1}{\PYZsq{}}\PY{l+s+s1}{black}\PY{l+s+s1}{\PYZsq{}}\PY{p}{,} \PY{n}{linewidth} \PY{o}{=} \PY{l+s+s1}{\PYZsq{}}\PY{l+s+s1}{5}\PY{l+s+s1}{\PYZsq{}}\PY{p}{)}
\PY{c+c1}{\PYZsh{} init\PYZus{}fig.text(cx, cy, df.columns[i], color=\PYZdq{}k\PYZdq{}, ha=\PYZdq{}center\PYZdq{}, va=\PYZdq{}center\PYZdq{})}

\PY{c+c1}{\PYZsh{}上のグラフのBAR内の番号記入}

\PY{k}{for} \PY{n}{i} \PY{o+ow}{in} \PY{n+nb}{range}\PY{p}{(}\PY{n+nb}{len}\PY{p}{(}\PY{n}{bar}\PY{p}{)}\PY{p}{)}\PY{p}{:}
        \PY{n}{cx} \PY{o}{=} \PY{n}{bar}\PY{p}{[}\PY{n}{i}\PY{p}{]}\PY{o}{.}\PY{n}{get\PYZus{}x}\PY{p}{(}\PY{p}{)} \PY{o}{+} \PY{n}{bar}\PY{p}{[}\PY{n}{i}\PY{p}{]}\PY{o}{.}\PY{n}{get\PYZus{}width}\PY{p}{(}\PY{p}{)} \PY{o}{/} \PY{l+m+mi}{2}
\PY{c+c1}{\PYZsh{}         print(cx)}
        \PY{n}{cy} \PY{o}{=} \PY{n}{bar}\PY{p}{[}\PY{n}{i}\PY{p}{]}\PY{o}{.}\PY{n}{get\PYZus{}y}\PY{p}{(}\PY{p}{)} \PY{o}{+} \PY{n}{bar}\PY{p}{[}\PY{n}{i}\PY{p}{]}\PY{o}{.}\PY{n}{get\PYZus{}height}\PY{p}{(}\PY{p}{)} \PY{o}{/} \PY{l+m+mi}{2}
\PY{c+c1}{\PYZsh{}         print(cy)}
        \PY{n}{init\PYZus{}fig}\PY{o}{.}\PY{n}{text}\PY{p}{(}\PY{n}{cx}\PY{p}{,} \PY{n}{cy}\PY{p}{,} \PY{n}{x\PYZus{}label}\PY{p}{[}\PY{n}{i}\PY{p}{]}\PY{p}{,} \PY{n}{size}\PY{o}{=} \PY{l+m+mi}{20}\PY{p}{,}  \PY{n}{color}\PY{o}{=}\PY{l+s+s2}{\PYZdq{}}\PY{l+s+s2}{k}\PY{l+s+s2}{\PYZdq{}}\PY{p}{,} \PY{n}{ha}\PY{o}{=}\PY{l+s+s2}{\PYZdq{}}\PY{l+s+s2}{center}\PY{l+s+s2}{\PYZdq{}}\PY{p}{,} \PY{n}{va}\PY{o}{=}\PY{l+s+s2}{\PYZdq{}}\PY{l+s+s2}{center}\PY{l+s+s2}{\PYZdq{}}\PY{p}{)}
        \PY{n}{init\PYZus{}fig}\PY{o}{.}\PY{n}{text}\PY{p}{(}\PY{n}{cx}\PY{p}{,} \PY{n}{cy}\PY{o}{\PYZhy{}}\PY{l+m+mf}{0.05}\PY{p}{,} \PY{n+nb}{str}\PY{p}{(}\PY{l+s+sa}{f}\PY{l+s+s1}{\PYZsq{}}\PY{l+s+si}{\PYZob{}}\PY{n}{prFill}\PY{p}{[}\PY{n}{i}\PY{p}{]}\PY{o}{*}\PY{l+m+mi}{100}\PY{l+s+si}{:}\PY{l+s+s1}{.0f}\PY{l+s+si}{\PYZcb{}}\PY{l+s+s1}{\PYZsq{}}\PY{p}{)} \PY{o}{+}\PY{l+s+s1}{\PYZsq{}}\PY{l+s+s1}{\PYZpc{}}\PY{l+s+s1}{\PYZsq{}}\PY{p}{,}\PY{n}{size}\PY{o}{=} \PY{l+m+mi}{20}\PY{p}{,}  \PY{n}{color}\PY{o}{=}\PY{l+s+s2}{\PYZdq{}}\PY{l+s+s2}{k}\PY{l+s+s2}{\PYZdq{}}\PY{p}{,} \PY{n}{ha}\PY{o}{=}\PY{l+s+s2}{\PYZdq{}}\PY{l+s+s2}{center}\PY{l+s+s2}{\PYZdq{}}\PY{p}{,} \PY{n}{va}\PY{o}{=}\PY{l+s+s2}{\PYZdq{}}\PY{l+s+s2}{center}\PY{l+s+s2}{\PYZdq{}}\PY{p}{)}
        
\PY{c+c1}{\PYZsh{}下のグラフの表示設定}
\PY{c+c1}{\PYZsh{}参考;https://www.yutaka\PYZhy{}note.com/entry/matplotlib\PYZus{}axis}

\PY{n}{stack\PYZus{}fig}\PY{o}{.}\PY{n}{set\PYZus{}xlabel}\PY{p}{(}\PY{l+s+s2}{\PYZdq{}}\PY{l+s+s2}{Line ID}\PY{l+s+s2}{\PYZdq{}}\PY{p}{,} \PY{n}{size} \PY{o}{=} \PY{l+m+mi}{25}\PY{p}{)}
\PY{n}{stack\PYZus{}fig}\PY{o}{.}\PY{n}{set\PYZus{}xticks}\PY{p}{(}\PY{n}{x\PYZus{}stack\PYZus{}label}\PY{p}{)}
\PY{n}{stack\PYZus{}fig}\PY{o}{.}\PY{n}{set\PYZus{}xticklabels}\PY{p}{(}\PY{n+nb}{list}\PY{p}{(}\PY{n+nb}{map}\PY{p}{(}\PY{k}{lambda} \PY{n}{label}\PY{p}{:}\PY{l+s+s1}{\PYZsq{}}\PY{l+s+s1}{L}\PY{l+s+s1}{\PYZsq{}} \PY{o}{+} \PY{n+nb}{str}\PY{p}{(}\PY{n}{label}\PY{p}{)}\PY{p}{,} \PY{n}{x\PYZus{}stack\PYZus{}label}\PY{p}{)}\PY{p}{)}\PY{p}{,} \PY{n}{size}\PY{o}{=}\PY{l+m+mi}{20}\PY{p}{)}

\PY{n}{stack\PYZus{}fig}\PY{o}{.}\PY{n}{set\PYZus{}ylabel}\PY{p}{(}\PY{l+s+s2}{\PYZdq{}}\PY{l+s+s2}{Action Steps(AS)}\PY{l+s+s2}{\PYZdq{}}\PY{p}{,} \PY{n}{size} \PY{o}{=} \PY{l+m+mi}{25}\PY{p}{)}
\PY{n}{stack\PYZus{}fig}\PY{o}{.}\PY{n}{set\PYZus{}yticks}\PY{p}{(}\PY{n}{y\PYZus{}label}\PY{p}{)}
\PY{n}{stack\PYZus{}fig}\PY{o}{.}\PY{n}{set\PYZus{}yticklabels}\PY{p}{(}\PY{n}{y\PYZus{}label}\PY{p}{,} \PY{n}{size}\PY{o}{=}\PY{l+m+mi}{20}\PY{p}{)}
\PY{n}{stack\PYZus{}fig}\PY{o}{.}\PY{n}{set\PYZus{}ylim}\PY{p}{(}\PY{l+m+mi}{0} \PY{p}{,} \PY{l+m+mi}{1}\PY{p}{)}
\PY{c+c1}{\PYZsh{} init\PYZus{}fig.set\PYZus{}yticks(np.arange(0, 1, 0.2))}
\PY{c+c1}{\PYZsh{} init\PYZus{}fig.title(\PYZdq{}\PYZdq{})}
\PY{n}{stack\PYZus{}fig}\PY{o}{.}\PY{n}{grid}\PY{p}{(}\PY{k+kc}{True}\PY{p}{)}
\PY{c+c1}{\PYZsh{}下のグラフの表示}
\PY{n}{bottom} \PY{o}{=} \PY{n}{np}\PY{o}{.}\PY{n}{zeros}\PY{p}{(}\PY{n}{stackBarD}\PY{o}{.}\PY{n}{T}\PY{o}{.}\PY{n}{shape}\PY{p}{[}\PY{l+m+mi}{1}\PY{p}{]}\PY{p}{)}

\PY{k}{for} \PY{n}{i} \PY{o+ow}{in} \PY{n+nb}{range}\PY{p}{(}\PY{n}{stackBarD}\PY{o}{.}\PY{n}{T}\PY{o}{.}\PY{n}{shape}\PY{p}{[}\PY{l+m+mi}{0}\PY{p}{]}\PY{p}{)}\PY{p}{:}
    
    \PY{k}{if} \PY{n}{i} \PY{o}{==}\PY{l+m+mi}{0}\PY{p}{:}
        \PY{n}{s\PYZus{}bar} \PY{o}{=} \PY{n}{stack\PYZus{}fig}\PY{o}{.}\PY{n}{bar}\PY{p}{(}\PY{n}{x\PYZus{}stack\PYZus{}label}\PY{p}{,} \PY{n}{stackBarD}\PY{o}{.}\PY{n}{T}\PY{p}{[}\PY{n}{i}\PY{p}{]}\PY{p}{,} \PY{n}{color} \PY{o}{=} \PY{l+s+s1}{\PYZsq{}}\PY{l+s+s1}{w}\PY{l+s+s1}{\PYZsq{}}\PY{p}{,} \PY{n}{edgecolor} \PY{o}{=}\PY{l+s+s1}{\PYZsq{}}\PY{l+s+s1}{black}\PY{l+s+s1}{\PYZsq{}}\PY{p}{,} \PY{n}{linewidth} \PY{o}{=} \PY{l+s+s1}{\PYZsq{}}\PY{l+s+s1}{5}\PY{l+s+s1}{\PYZsq{}}\PY{p}{)}
    \PY{k}{else}\PY{p}{:}
        \PY{n}{s\PYZus{}bar} \PY{o}{=} \PY{n}{stack\PYZus{}fig}\PY{o}{.}\PY{n}{bar}\PY{p}{(}\PY{n}{x\PYZus{}stack\PYZus{}label}\PY{p}{,} \PY{n}{stackBarD}\PY{o}{.}\PY{n}{T}\PY{p}{[}\PY{n}{i}\PY{p}{]}\PY{p}{,} \PY{n}{bottom}\PY{o}{=} \PY{n}{bottom}\PY{p}{,} \PY{n}{color} \PY{o}{=} \PY{l+s+s1}{\PYZsq{}}\PY{l+s+s1}{w}\PY{l+s+s1}{\PYZsq{}}\PY{p}{,} \PY{n}{edgecolor} \PY{o}{=}\PY{l+s+s1}{\PYZsq{}}\PY{l+s+s1}{black}\PY{l+s+s1}{\PYZsq{}}\PY{p}{,} \PY{n}{linewidth} \PY{o}{=} \PY{l+s+s1}{\PYZsq{}}\PY{l+s+s1}{5}\PY{l+s+s1}{\PYZsq{}}\PY{p}{)}
        
    
    \PY{n}{bottom} \PY{o}{=} \PY{n}{np}\PY{o}{.}\PY{n}{add}\PY{p}{(}\PY{n}{bottom}\PY{p}{,} \PY{n}{stackBarD}\PY{o}{.}\PY{n}{T}\PY{p}{[}\PY{n}{i}\PY{p}{]}\PY{p}{)}
    \PY{n+nb}{print}\PY{p}{(}\PY{n}{bottom}\PY{p}{)}
\PY{c+c1}{\PYZsh{} stack\PYZus{}fig.bar(x\PYZus{}stack\PYZus{}label, stackBarD.T[0], color = \PYZsq{}w\PYZsq{}, edgecolor =\PYZsq{}black\PYZsq{}, linewidth = \PYZsq{}5\PYZsq{})}
\PY{c+c1}{\PYZsh{} stack\PYZus{}fig.bar(x\PYZus{}stack\PYZus{}label, stackBarD.T[1], bottom= stackBarD.T[0], color = \PYZsq{}w\PYZsq{}, edgecolor =\PYZsq{}black\PYZsq{}, linewidth = \PYZsq{}5\PYZsq{})}
\PY{c+c1}{\PYZsh{} stack\PYZus{}fig.bar(x\PYZus{}stack\PYZus{}label, stackBarD.T[2], bottom= stackBarD.T[0] + stackBarD.T[1], color = \PYZsq{}w\PYZsq{}, edgecolor =\PYZsq{}black\PYZsq{}, linewidth = \PYZsq{}5\PYZsq{})}
\PY{n}{n\PYZus{}LineT} \PY{o}{=} \PY{n+nb}{list}\PY{p}{(}\PY{n+nb}{map}\PY{p}{(}\PY{k}{lambda} \PY{n}{Line}\PY{p}{:}\PY{n}{Line} \PY{o}{+}\PY{l+m+mi}{1}\PY{p}{,} \PY{n}{f\PYZus{}LineT}\PY{p}{)}\PY{p}{)}

\PY{k}{for} \PY{n}{i} \PY{o+ow}{in} \PY{n+nb}{range}\PY{p}{(}\PY{n}{stackBarD}\PY{o}{.}\PY{n}{shape}\PY{p}{[}\PY{l+m+mi}{0}\PY{p}{]}\PY{p}{)}\PY{p}{:}
    \PY{n}{baseY}\PY{o}{=}\PY{l+m+mi}{0}
    \PY{k}{for} \PY{n}{j} \PY{o+ow}{in} \PY{n+nb}{range}\PY{p}{(}\PY{n}{stackBarD}\PY{o}{.}\PY{n}{shape}\PY{p}{[}\PY{l+m+mi}{1}\PY{p}{]}\PY{p}{)}\PY{p}{:}
        \PY{k}{if} \PY{n}{stackBarD}\PY{p}{[}\PY{n}{i}\PY{p}{]}\PY{p}{[}\PY{n}{j}\PY{p}{]}\PY{o}{\PYZgt{}}\PY{l+m+mi}{0}\PY{p}{:}
\PY{c+c1}{\PYZsh{}             print(\PYZsq{}ffff\PYZsq{},f\PYZus{}LineT[i].size)}
            \PY{k}{if} \PY{n}{f\PYZus{}LineT}\PY{p}{[}\PY{n}{i}\PY{p}{]}\PY{o}{.}\PY{n}{size} \PY{o}{==}\PY{l+m+mi}{1}\PY{p}{:}
                \PY{n}{key} \PY{o}{=} \PY{n}{f\PYZus{}LineT}\PY{p}{[}\PY{n}{i}\PY{p}{]}
                \PY{n+nb}{print}\PY{p}{(}\PY{l+s+s1}{\PYZsq{}}\PY{l+s+s1}{key1;}\PY{l+s+s1}{\PYZsq{}}\PY{p}{,}\PY{n}{key}\PY{p}{)}
                \PY{n}{tmp} \PY{o}{=} \PY{n}{prFill}\PY{p}{[}\PY{n}{key}\PY{p}{]}
                \PY{n+nb}{print}\PY{p}{(}\PY{l+s+s1}{\PYZsq{}}\PY{l+s+s1}{tmp1;}\PY{l+s+s1}{\PYZsq{}}\PY{p}{,}\PY{n}{tmp}\PY{p}{)}
            \PY{k}{else}\PY{p}{:}
                \PY{n}{key} \PY{o}{=} \PY{n}{f\PYZus{}LineT}\PY{p}{[}\PY{n}{i}\PY{p}{]}\PY{p}{[}\PY{n}{j}\PY{p}{]}
                \PY{n+nb}{print}\PY{p}{(}\PY{l+s+s1}{\PYZsq{}}\PY{l+s+s1}{key;}\PY{l+s+s1}{\PYZsq{}}\PY{p}{,}\PY{n}{key}\PY{p}{)}
                \PY{n}{tmp} \PY{o}{=} \PY{n}{prFill}\PY{p}{[}\PY{n}{key}\PY{p}{]}
                \PY{n+nb}{print}\PY{p}{(}\PY{l+s+s1}{\PYZsq{}}\PY{l+s+s1}{tmp;}\PY{l+s+s1}{\PYZsq{}}\PY{p}{,}\PY{n}{tmp}\PY{p}{)}
            \PY{n}{ypos} \PY{o}{=} \PY{n}{tmp}\PY{o}{/}\PY{l+m+mi}{2}
            \PY{n}{stack\PYZus{}fig}\PY{o}{.}\PY{n}{text}\PY{p}{(}\PY{n}{s\PYZus{}bar}\PY{p}{[}\PY{n}{i}\PY{p}{]}\PY{o}{.}\PY{n}{get\PYZus{}x}\PY{p}{(}\PY{p}{)} \PY{o}{+} \PY{n}{s\PYZus{}bar}\PY{p}{[}\PY{n}{i}\PY{p}{]}\PY{o}{.}\PY{n}{get\PYZus{}width}\PY{p}{(}\PY{p}{)} \PY{o}{/} \PY{l+m+mi}{2}\PY{p}{,} \PY{n}{baseY}\PY{o}{+} \PY{n}{ypos}\PY{p}{,} \PY{n+nb}{str}\PY{p}{(}\PY{n}{key}\PY{o}{+}\PY{l+m+mi}{1}\PY{p}{)}\PY{p}{,}\PY{n}{size}\PY{o}{=} \PY{l+m+mi}{20}\PY{p}{,}  \PY{n}{color}\PY{o}{=}\PY{l+s+s2}{\PYZdq{}}\PY{l+s+s2}{k}\PY{l+s+s2}{\PYZdq{}}\PY{p}{,} \PY{n}{ha}\PY{o}{=}\PY{l+s+s2}{\PYZdq{}}\PY{l+s+s2}{center}\PY{l+s+s2}{\PYZdq{}}\PY{p}{,} \PY{n}{va}\PY{o}{=}\PY{l+s+s2}{\PYZdq{}}\PY{l+s+s2}{center}\PY{l+s+s2}{\PYZdq{}}\PY{p}{)}
            \PY{n}{stack\PYZus{}fig}\PY{o}{.}\PY{n}{text}\PY{p}{(}\PY{n}{s\PYZus{}bar}\PY{p}{[}\PY{n}{i}\PY{p}{]}\PY{o}{.}\PY{n}{get\PYZus{}x}\PY{p}{(}\PY{p}{)} \PY{o}{+} \PY{n}{s\PYZus{}bar}\PY{p}{[}\PY{n}{i}\PY{p}{]}\PY{o}{.}\PY{n}{get\PYZus{}width}\PY{p}{(}\PY{p}{)} \PY{o}{/} \PY{l+m+mi}{2}\PY{p}{,} \PY{n}{baseY}\PY{o}{+} \PY{n}{ypos} \PY{o}{\PYZhy{}} \PY{l+m+mf}{0.05}\PY{p}{,} \PY{n+nb}{str}\PY{p}{(}\PY{l+s+sa}{f}\PY{l+s+s1}{\PYZsq{}}\PY{l+s+si}{\PYZob{}}\PY{n}{tmp}\PY{o}{*}\PY{l+m+mi}{100}\PY{l+s+si}{:}\PY{l+s+s1}{.0f}\PY{l+s+si}{\PYZcb{}}\PY{l+s+s1}{\PYZsq{}}\PY{p}{)} \PY{o}{+}\PY{l+s+s1}{\PYZsq{}}\PY{l+s+s1}{\PYZpc{}}\PY{l+s+s1}{\PYZsq{}}\PY{p}{,}\PY{n}{size}\PY{o}{=} \PY{l+m+mi}{20}\PY{p}{,}  \PY{n}{color}\PY{o}{=}\PY{l+s+s2}{\PYZdq{}}\PY{l+s+s2}{k}\PY{l+s+s2}{\PYZdq{}}\PY{p}{,} \PY{n}{ha}\PY{o}{=}\PY{l+s+s2}{\PYZdq{}}\PY{l+s+s2}{center}\PY{l+s+s2}{\PYZdq{}}\PY{p}{,} \PY{n}{va}\PY{o}{=}\PY{l+s+s2}{\PYZdq{}}\PY{l+s+s2}{center}\PY{l+s+s2}{\PYZdq{}}\PY{p}{)}
            \PY{n}{baseY} \PY{o}{=} \PY{n}{baseY}\PY{o}{+}\PY{n}{tmp}
\PY{c+c1}{\PYZsh{}     cx = s\PYZus{}bar[j].get\PYZus{}x() + s\PYZus{}bar[j].get\PYZus{}width() / 2}
\PY{c+c1}{\PYZsh{}     print(cx)}
\PY{c+c1}{\PYZsh{}     cy = s\PYZus{}bar[j].get\PYZus{}y() + s\PYZus{}bar[j].get\PYZus{}height() / 2}
\PY{c+c1}{\PYZsh{}     print(cy)}
\PY{c+c1}{\PYZsh{}     stack\PYZus{}fig.text(cx, cy,stackBarD.T[0][j], size= 20,  color=\PYZdq{}k\PYZdq{}, ha=\PYZdq{}center\PYZdq{}, va=\PYZdq{}center\PYZdq{})}
\PY{c+c1}{\PYZsh{} stack\PYZus{}fig}
\PY{c+c1}{\PYZsh{}       init\PYZus{}fig.text(cx, cy\PYZhy{}0.05, str(f\PYZsq{}\PYZob{}prFill[i]*100:.0f\PYZcb{}\PYZsq{}) +\PYZsq{}\PYZpc{}\PYZsq{},size= 20,  color=\PYZdq{}k\PYZdq{}, ha=\PYZdq{}center\PYZdq{}, va=\PYZdq{}center\PYZdq{})}
\end{Verbatim}
\end{tcolorbox}

    \begin{Verbatim}[commandchars=\\\{\}]
[0.9 0.6 0.5 0.5 0.9 0.8 0.4]
[0.9 1.  1.  0.8 0.9 0.8 0.4]
[0.9 1.  1.  1.  0.9 0.8 0.4]
key1; 0
tmp1; 0.9
key; 1
tmp; 0.6
key; 6
tmp; 0.4
key; 2
tmp; 0.5
key; 3
tmp; 0.5
key; 4
tmp; 0.5
key; 7
tmp; 0.3
key; 10
tmp; 0.2
key1; 5
tmp1; 0.9
key1; 8
tmp1; 0.8
key1; 9
tmp1; 0.4
    \end{Verbatim}

    \begin{center}
    \adjustimage{max size={0.9\linewidth}{0.9\paperheight}}{lec4_step7_BarStack_Aligned_Stage6_files/lec4_step7_BarStack_Aligned_Stage6_15_1.png}
    \end{center}
    { \hspace*{\fill} \\}
    
    \begin{tcolorbox}[breakable, size=fbox, boxrule=1pt, pad at break*=1mm,colback=cellbackground, colframe=cellborder]
\prompt{In}{incolor}{317}{\boxspacing}
\begin{Verbatim}[commandchars=\\\{\}]
\PY{n+nb}{print}\PY{p}{(}\PY{n}{stackBarD}\PY{p}{)}
\PY{n+nb}{print}\PY{p}{(}\PY{n}{f\PYZus{}LineT}\PY{p}{)}
\PY{n+nb}{print}\PY{p}{(}\PY{n}{n\PYZus{}LineT}\PY{p}{)}
\PY{n+nb}{print}\PY{p}{(}\PY{n}{prFill}\PY{p}{)}
\PY{n}{f\PYZus{}LineT}\PY{p}{[}\PY{l+m+mi}{0}\PY{p}{]}\PY{o}{.}\PY{n}{size}
\end{Verbatim}
\end{tcolorbox}

    \begin{Verbatim}[commandchars=\\\{\}]
[[0.9 0.  0. ]
 [0.6 0.4 0. ]
 [0.5 0.5 0. ]
 [0.5 0.3 0.2]
 [0.9 0.  0. ]
 [0.8 0.  0. ]
 [0.4 0.  0. ]]
[array(0), array([1, 6], dtype=int64), array([2, 3], dtype=int64), array([ 4,
7, 10], dtype=int64), array(5), array(8), array(9)]
[1, array([2, 7], dtype=int64), array([3, 4], dtype=int64), array([ 5,  8, 11],
dtype=int64), 6, 9, 10]
[0.9 0.6 0.5 0.5 0.5 0.9 0.4 0.3 0.8 0.4 0.2]
    \end{Verbatim}

            \begin{tcolorbox}[breakable, size=fbox, boxrule=.5pt, pad at break*=1mm, opacityfill=0]
\prompt{Out}{outcolor}{317}{\boxspacing}
\begin{Verbatim}[commandchars=\\\{\}]
1
\end{Verbatim}
\end{tcolorbox}
        
    \begin{tcolorbox}[breakable, size=fbox, boxrule=1pt, pad at break*=1mm,colback=cellbackground, colframe=cellborder]
\prompt{In}{incolor}{268}{\boxspacing}
\begin{Verbatim}[commandchars=\\\{\}]
\PY{n+nb}{print}\PY{p}{(}\PY{n}{LineT}\PY{p}{)}
\PY{n+nb}{print}\PY{p}{(}\PY{n+nb}{type}\PY{p}{(}\PY{n}{LineT}\PY{p}{)}\PY{p}{)}
\PY{n}{n\PYZus{}LineT} \PY{o}{=} \PY{n+nb}{list}\PY{p}{(}\PY{n+nb}{map}\PY{p}{(}\PY{k}{lambda} \PY{n}{Line}\PY{p}{:}\PY{n}{Line} \PY{o}{+}\PY{l+m+mi}{1}\PY{p}{,} \PY{n}{LineT}\PY{p}{)}\PY{p}{)}
\PY{n+nb}{print}\PY{p}{(}\PY{n}{l}\PY{p}{)}
\PY{n+nb}{print}\PY{p}{(}\PY{n+nb}{type}\PY{p}{(}\PY{n}{l}\PY{p}{)}\PY{p}{)}
\end{Verbatim}
\end{tcolorbox}

    \begin{Verbatim}[commandchars=\\\{\}]
[0, array([1, 6], dtype=int64), array([2, 3], dtype=int64), array([ 4,  7, 10],
dtype=int64), 5, 8, 9]
<class 'list'>
[1, array([2, 7], dtype=int64), array([3, 4], dtype=int64), array([ 5,  8, 11],
dtype=int64), 6, 9, 10]
<class 'list'>
    \end{Verbatim}

    \begin{tcolorbox}[breakable, size=fbox, boxrule=1pt, pad at break*=1mm,colback=cellbackground, colframe=cellborder]
\prompt{In}{incolor}{269}{\boxspacing}
\begin{Verbatim}[commandchars=\\\{\}]
\PY{n+nb}{print}\PY{p}{(}\PY{n}{LineT}\PY{p}{)}
\PY{n+nb}{print}\PY{p}{(}\PY{n}{stackBarD}\PY{o}{.}\PY{n}{T}\PY{p}{)}
\end{Verbatim}
\end{tcolorbox}

    \begin{Verbatim}[commandchars=\\\{\}]
[0, array([1, 6], dtype=int64), array([2, 3], dtype=int64), array([ 4,  7, 10],
dtype=int64), 5, 8, 9]
[[0.9 0.6 0.5 0.5 0.9 0.8 0.4]
 [0.  0.4 0.5 0.3 0.  0.  0. ]
 [0.  0.  0.  0.2 0.  0.  0. ]]
    \end{Verbatim}

    \begin{tcolorbox}[breakable, size=fbox, boxrule=1pt, pad at break*=1mm,colback=cellbackground, colframe=cellborder]
\prompt{In}{incolor}{270}{\boxspacing}
\begin{Verbatim}[commandchars=\\\{\}]
\PY{c+c1}{\PYZsh{}下のグラフの表示設定}
\PY{c+c1}{\PYZsh{}参考;https://www.yutaka\PYZhy{}note.com/entry/matplotlib\PYZus{}axis}

\PY{n}{stack\PYZus{}fig}\PY{o}{.}\PY{n}{set\PYZus{}xlabel}\PY{p}{(}\PY{l+s+s2}{\PYZdq{}}\PY{l+s+s2}{store ID}\PY{l+s+s2}{\PYZdq{}}\PY{p}{,} \PY{n}{size} \PY{o}{=} \PY{l+m+mi}{25}\PY{p}{)}
\PY{n}{stack\PYZus{}fig}\PY{o}{.}\PY{n}{set\PYZus{}xticks}\PY{p}{(}\PY{n}{x\PYZus{}stack\PYZus{}label}\PY{p}{)}
\PY{n}{stack\PYZus{}fig}\PY{o}{.}\PY{n}{set\PYZus{}xticklabels}\PY{p}{(}\PY{n}{x\PYZus{}stack\PYZus{}label}\PY{p}{,} \PY{n}{size}\PY{o}{=}\PY{l+m+mi}{20}\PY{p}{)}

\PY{n}{stack\PYZus{}fig}\PY{o}{.}\PY{n}{set\PYZus{}ylabel}\PY{p}{(}\PY{l+s+s2}{\PYZdq{}}\PY{l+s+s2}{Action Steps(AS)}\PY{l+s+s2}{\PYZdq{}}\PY{p}{,} \PY{n}{size} \PY{o}{=} \PY{l+m+mi}{25}\PY{p}{)}
\PY{n}{stack\PYZus{}fig}\PY{o}{.}\PY{n}{set\PYZus{}yticks}\PY{p}{(}\PY{n}{y\PYZus{}label}\PY{p}{)}
\PY{n}{stack\PYZus{}fig}\PY{o}{.}\PY{n}{set\PYZus{}yticklabels}\PY{p}{(}\PY{n}{y\PYZus{}label}\PY{p}{,} \PY{n}{size}\PY{o}{=}\PY{l+m+mi}{20}\PY{p}{)}
\PY{n}{stack\PYZus{}fig}\PY{o}{.}\PY{n}{set\PYZus{}ylim}\PY{p}{(}\PY{l+m+mi}{0} \PY{p}{,} \PY{l+m+mi}{1}\PY{p}{)}
\PY{c+c1}{\PYZsh{} init\PYZus{}fig.set\PYZus{}yticks(np.arange(0, 1, 0.2))}
\PY{c+c1}{\PYZsh{} init\PYZus{}fig.title(\PYZdq{}\PYZdq{})}
\PY{n}{stack\PYZus{}fig}\PY{o}{.}\PY{n}{grid}\PY{p}{(}\PY{k+kc}{True}\PY{p}{)}
\PY{c+c1}{\PYZsh{}上のグラフの表示}
\PY{n}{stack\PYZus{}fig}\PY{o}{.}\PY{n}{bar}\PY{p}{(}\PY{n}{x\PYZus{}stack\PYZus{}label}\PY{p}{,} \PY{n}{stackBarD}\PY{o}{.}\PY{n}{T}\PY{p}{[}\PY{l+m+mi}{0}\PY{p}{]}\PY{p}{,} \PY{n}{color} \PY{o}{=} \PY{l+s+s1}{\PYZsq{}}\PY{l+s+s1}{w}\PY{l+s+s1}{\PYZsq{}}\PY{p}{,} \PY{n}{edgecolor} \PY{o}{=}\PY{l+s+s1}{\PYZsq{}}\PY{l+s+s1}{black}\PY{l+s+s1}{\PYZsq{}}\PY{p}{,} \PY{n}{linewidth} \PY{o}{=} \PY{l+s+s1}{\PYZsq{}}\PY{l+s+s1}{5}\PY{l+s+s1}{\PYZsq{}}\PY{p}{)}
\end{Verbatim}
\end{tcolorbox}

            \begin{tcolorbox}[breakable, size=fbox, boxrule=.5pt, pad at break*=1mm, opacityfill=0]
\prompt{Out}{outcolor}{270}{\boxspacing}
\begin{Verbatim}[commandchars=\\\{\}]
<BarContainer object of 7 artists>
\end{Verbatim}
\end{tcolorbox}
        
    \begin{tcolorbox}[breakable, size=fbox, boxrule=1pt, pad at break*=1mm,colback=cellbackground, colframe=cellborder]
\prompt{In}{incolor}{26}{\boxspacing}
\begin{Verbatim}[commandchars=\\\{\}]
\PY{n}{init\PYZus{}fig}\PY{o}{.}\PY{n}{bar}\PY{p}{(}\PY{n}{x\PYZus{}label}\PY{p}{,} \PY{n}{prFill}\PY{p}{)}
\PY{c+c1}{\PYZsh{} plt.show}
\end{Verbatim}
\end{tcolorbox}

            \begin{tcolorbox}[breakable, size=fbox, boxrule=.5pt, pad at break*=1mm, opacityfill=0]
\prompt{Out}{outcolor}{26}{\boxspacing}
\begin{Verbatim}[commandchars=\\\{\}]
<BarContainer object of 11 artists>
\end{Verbatim}
\end{tcolorbox}
        
    \begin{tcolorbox}[breakable, size=fbox, boxrule=1pt, pad at break*=1mm,colback=cellbackground, colframe=cellborder]
\prompt{In}{incolor}{ }{\boxspacing}
\begin{Verbatim}[commandchars=\\\{\}]
\PY{n+nb}{len}\PY{p}{(}\PY{n}{LineT}\PY{p}{[}\PY{l+m+mi}{1}\PY{p}{]}\PY{p}{)}
\end{Verbatim}
\end{tcolorbox}

    \begin{tcolorbox}[breakable, size=fbox, boxrule=1pt, pad at break*=1mm,colback=cellbackground, colframe=cellborder]
\prompt{In}{incolor}{ }{\boxspacing}
\begin{Verbatim}[commandchars=\\\{\}]
\PY{n}{LineT}\PY{p}{[}\PY{l+m+mi}{2}\PY{p}{]}\PY{o}{.}\PY{n}{size}
\end{Verbatim}
\end{tcolorbox}

    \begin{tcolorbox}[breakable, size=fbox, boxrule=1pt, pad at break*=1mm,colback=cellbackground, colframe=cellborder]
\prompt{In}{incolor}{140}{\boxspacing}
\begin{Verbatim}[commandchars=\\\{\}]
\PY{n}{np}\PY{o}{.}\PY{n}{zeros}\PY{p}{(}\PY{n}{stackBarD}\PY{o}{.}\PY{n}{T}\PY{o}{.}\PY{n}{shape}\PY{p}{[}\PY{l+m+mi}{1}\PY{p}{]}\PY{p}{)}
\end{Verbatim}
\end{tcolorbox}

            \begin{tcolorbox}[breakable, size=fbox, boxrule=.5pt, pad at break*=1mm, opacityfill=0]
\prompt{Out}{outcolor}{140}{\boxspacing}
\begin{Verbatim}[commandchars=\\\{\}]
array([0., 0., 0., 0., 0., 0., 0.])
\end{Verbatim}
\end{tcolorbox}
        
    \begin{tcolorbox}[breakable, size=fbox, boxrule=1pt, pad at break*=1mm,colback=cellbackground, colframe=cellborder]
\prompt{In}{incolor}{ }{\boxspacing}
\begin{Verbatim}[commandchars=\\\{\}]

\end{Verbatim}
\end{tcolorbox}

    \begin{tcolorbox}[breakable, size=fbox, boxrule=1pt, pad at break*=1mm,colback=cellbackground, colframe=cellborder]
\prompt{In}{incolor}{ }{\boxspacing}
\begin{Verbatim}[commandchars=\\\{\}]

\end{Verbatim}
\end{tcolorbox}

    \begin{tcolorbox}[breakable, size=fbox, boxrule=1pt, pad at break*=1mm,colback=cellbackground, colframe=cellborder]
\prompt{In}{incolor}{ }{\boxspacing}
\begin{Verbatim}[commandchars=\\\{\}]

\end{Verbatim}
\end{tcolorbox}

    \begin{tcolorbox}[breakable, size=fbox, boxrule=1pt, pad at break*=1mm,colback=cellbackground, colframe=cellborder]
\prompt{In}{incolor}{ }{\boxspacing}
\begin{Verbatim}[commandchars=\\\{\}]

\end{Verbatim}
\end{tcolorbox}


    % Add a bibliography block to the postdoc
    
    
    
\end{document}
